\documentclass[../main.tex]{subfiles}

\begin{document}
		
	\begin{definition}[Topologia]
		Sia $X$ un insieme non vuoto e sia $\tau$ una famiglia di sottoinsiemi di $X$. La coppia $(X,\tau)$ si dice ``spazio topologico'' e $\tau$ si dice ``topologia'' se:
		\begin{enumerate}
			\item $X\in\tau$ e $\varnothing\in\tau$;
			\item $\suc{A}{i}{I}\subseteq\tau \implies \bigcup_{i\in I} A_i \in \tau$;
			\item $\ennu{A}{n}\subseteq\tau \implies \bigcap_{i=1}^n A_i \in \tau$.
		\end{enumerate}
		Gli elementi di $\tau$ vengono detti ``aperti'', e i loro complementari vengono detti ``chiusi''.
	\end{definition}
	
	\begin{nota}
		Quando non è necessario specificare la topologia $\tau$, la coppia $(X,\tau)$ si indica solo con $X$.
	\end{nota}
	
	\begin{definition}[Funzione continua]
		Siano $(X,\tau)$ e $(Y,\gamma)$ spazi topologici. Una funzione $f : X\to Y$ si dice ``continua'' se $\forall V\in\gamma$ si ha $f^{-1}(V)\in\tau$. La funzione $f$ si dice continua nel punto $x_0\in X$ se per ogni intorno $V$ di $f(x_0)$ esiste un intorno $U$ di $x_0$ tale che $f(U)\subseteq V$.
	\end{definition}
	
	\begin{exercise}
		Una funzione $f$ tra spazi topologici è continua se e solo se è continua in ogni punto $x_0\in X$.
	\end{exercise}
	
	\begin{definition}[$\sigma$-algebra]
		Sia $X$ un insieme non vuoto e sia $\mathscr{M}$ una famiglia di sottoinsiemi di $X$. $\mathscr{M}$ si dice ``$\sigma$-algebra'' se:
		\begin{enumerate}
			\item $X\in\mathscr{M}$;
			\item $A\in\mathscr{M} \implies A^C\in\mathscr{M}$;
			\item $\suc{A}{i}{\N}\in\mathscr{M} \implies \bigcup_{i\in \N} A_i \in \mathscr{M}$.
		\end{enumerate}
		La coppia $(X,\mathscr{M})$ è detta ``spazio misurabile'', e gli elementi di $\mathscr{M}$ sono detti ``insiemi misurabili''.
	\end{definition}
	
	\begin{oss}
		È immediato dedurre che le $\sigma$-algebre contengono le intersezioni numerabili dei propri elementi dato che
		\begin{equation*}
			\bigcap_{i\in I} A_i = \left(\bigcup_{i\in I} A_i^C \right)^C \,.
		\end{equation*}
	\end{oss}
	
	\begin{definition}[Funzione misurabile]
		Dati uno spazio misurabile $(X,\mathscr{M})$ e uno spazio topologico $(Y,\gamma)$, una funzione $f : X\to Y$ si dice ``misurabile'' se $\forall V\in \gamma$ si ha $f^{-1}(V)\in\mathscr{M}$.
	\end{definition}
	
	\begin{definition}[Spazio metrico]
		Sia $X$ un insieme non vuoto. La funzione $d : X\times X \to \R^+$ si dice ``distanza'' o ``metrica'' se:
		\begin{enumerate}
			\item $\forall x,y\in X \quad d(x,y)=d(y,x)$;
			\item $d(x,y)=0 \iff x=y$;
			\item $\forall x,y,z\in X \quad d(x,y)\leq d(x,z)+d(z,y)$.
		\end{enumerate}
		La coppia $(X,d)$ è detta ``spazio metrico''.
	\end{definition}
	
	\begin{nota}
		Dato uno spazio metrico $(X,d)$, consideriamo gli insiemi della forma
		\begin{equation*}
			B_\eps (x_0) = \left\{ x\in X \,\middle\vert\, d(x,x_0)<\eps \right\}
		\end{equation*}
		per ogni $\eps>0$ e $x_0\in X$, detti palle aperte di raggio $\eps$ centrate in $x_0$. Si dice ``topologia indotta'' su $X$ da $d$ e si indica con $\tau_d$ la famiglia di unioni qualsiasi e intersezioni finite di palle.
	\end{nota}
	
	\begin{prop}
		Una $\sigma$-algebra è finita o più che numerabile.
	\end{prop}
	\begin{proof}
		Per assurdo sia $\#\mathscr{M}=\aleph_0$, che scriviamo come $\mathscr{M}=\suc{A}{n}{\N}$ insiemi distinti. Per ogni $x\in X$, definiamo l'insieme
		\begin{equation*}
			B_x = \bigcap_{x\in A\in \mathscr{M}} A \,,
		\end{equation*}
		ossia l'intersezione dei misurabili contenenti $x$. Osservato che $B_x$ è il più piccolo misurabile contenente $x$, si dimostra facilmente che $\suc{B}{x}{X}$ è una partizione di $X$. Infatti, se presi $x,y\in X$ supponiamo che $B_x\cap B_y\ne \emptyset$, troviamo $z\in B_x\cap B_y$. Allora $B_z\subseteq B_x\cap B_y$. Se per assurdo $x\notin B_z$ allora $x\in B_x\setminus B_z$ che sarebbe un misurabile strettamente contenuto in $B_x$ contro la minimalità di quest'ultimo; analogamente se $y\notin B_z$. Concludiamo che
		\begin{align*}
			B_x \subseteq B_z \subseteq B_x \cap B_y \subseteq B_x \\
			B_y \subseteq B_z \subseteq B_x \cap B_y \subseteq B_y
		\end{align*}
		da cui $B_x=B_y=B_z$. Chiamiamo quindi $\mathscr{B}=\suc{B}{x}{X}$, e mostriamo che $\#\mathscr{B}=\aleph_0$.
		\begin{enumerate}
			\item Non può essere $\#\mathscr{B}>\aleph_0$ perché, dal momento che $\mathscr{B}\subseteq\mathscr{M}$, si avrebbe $\aleph_0 < \#\mathscr{B} \leq \#\mathscr{M} = \aleph_0$ che è assurdo;
			\item Non può essere $\#\mathscr{B}=n\in\N$ perché, dal momento che ogni $A\in\mathscr{M}$ si scrive come $A=\bigcup_{x\in A} B_x$, avremmo che $\#\mathscr{M} \leq 2^n$, che è assurdo.
		\end{enumerate}
		Scritto quindi $\mathscr{B}=\suc{B}{{x_n}}{\N}$, sia $\mathscr{C}$ l'insieme di tutte le unioni finite o numerabili di elementi di $\mathscr{B}$, e consideriamo la funzione
		\begin{equation*}
			\Phi : I \in \mathscr{P}(\N) \mapsto \bigcup_{i\in I} B_{x_i} \in \mathscr{C} \,,
		\end{equation*}
		che risulta essere iniettiva. Infatti, sapendo che $B_{x_i}=B_{x_j}$ implica $i=j$, se abbiamo $I\ne J\subseteq \mathscr{P}(\N)$ allora sicuramente $\Phi(I)\ne\Phi(J)$. Allora concludiamo che
		\begin{equation*}
			\aleph_0 < \mathscr{P}(\N) \leq \#\mathscr{C} \leq \mathscr{M} = \aleph_0 \,,
		\end{equation*}
		che è assurdo.	
	\end{proof}
	
	\begin{prop}[Composizione funzioni continue su spazi topologici]
		Siano $ (X,\tau _X), (Y,\tau _Y)$ e $ (Z,\tau_Z)$ spazi topologici, e siano $f: X\to Y$ e $g: Y\to Z$ funzioni, se $f$ e $g$ sono continue allora $g \circ f$ è continua.
	\end{prop}
	\begin{proof}
		Sia $W\in\tau_Z$. Per continuità di $g$ si ha $g^{-1}(W)\in\tau_Y$ e per continuità di $f$ si conclude che $f^{-1}(g^{-1}(W))=(g\circ f)^{-1}(W)\in\tau_X$.
	\end{proof}
	
	\begin{prop}[Composizione di funzioni su spazi topologici e misurabili]
		Siano $ (X,\tau_X, \mathscr{M}_X)$, $(Y,\tau_Y, \mathscr{M}_Y)$, $(Z,\tau_Z, \mathscr{M}_Z) $ spazi topologici misurabili, siano le funzioni  $f: X\to Y$ misurabile e $g:Y\to Z$ continua, allora $g \circ f$ è misurabile.
	\end{prop}
	\begin{proof}
		Sia $W\in\tau_Z$. Per continuità di $g$ si ha $g^{-1}(W)\in\tau_Y$ e per misurabilità di $f$ si conclude che $f^{-1}(g^{-1}(W))=(g\circ f)^{-1}(W)\in\mathscr{M}_X$.
	\end{proof}
	
	\begin{prop}[Composizione di funzioni misurabili]
		Siano $u,v: X \to \R$ funzioni reali misurabili, sia $\Phi: \R^2\to Y$ continua; allora $\Phi(u(x),v(x))$ è misurabile.
	\end{prop}
	\begin{proof}
		Definiamo $f : x\in X \mapsto (u(x),v(x)) \in \R^2$. Dimostriamo che $\Phi\circ f$ è misurabile. Sia $V\subseteq Y$ aperto e sia $W\defeq \Phi^{-1}(Y)$ che per continuità è un aperto di $\R^2$. Possiamo quindi dire che
		\begin{equation*}
			W = \bigcup_{n\in\N} R_n \,,
		\end{equation*}
		dove gli $R_n$ sono rettangoli aperti. Dunque, abbiamo che
		\begin{equation*}
			(\Phi\circ f)^{-1}(V) = f^{-1}(\Phi^{-1}(V)) = f^{-1}(W) = f^{-1}\left( \bigcup_{n\in\N} R_n \right) = \bigcup_{n\in\N} f^{-1}(R_n) \,.
		\end{equation*}
		Un qualsiasi rettangolo si scrive come prodotto di due intervalli aperti, e l'antimmagine mediante $f$ si trova calcolando l'antimmagine mediante $u$ del primo intervallo aperto e l'antimmagine mediante $v$ del secondo. Essendo $u,v$ misurabili, anche $f^{-1}(R_n)$ è misurabile, da cui la misurabilità della composta.
	\end{proof}
	
	\begin{corol}[Parte reale e parte immaginaria]
		Una funzione $f : X \to \C$ è misurabile se e solo se sono misurabili la sua parte reale e la sua parte immaginaria.
	\end{corol}
	\begin{proof}
		Dette $u : x\in X \mapsto \Re(f(x)) \in \R$ e $v : x\in X \mapsto \Im(f(x)) \in \R$, e detta $\Phi : (x,y) \in \R^2 \mapsto x+iy \in \C$, si trova per il teorema precedente l'implicazione ($\impliedby$).
		
		Supponendo $f$ misurabile, le funzioni parte reale e parte immaginaria sono continue. Dunque, la loro composizione è misurabile.
	\end{proof}
	
	\begin{oss}[Modulo, somma, prodotto]
		Usando i teoremi di composizione si mostrano facilmente le seguenti proprietà.
		\begin{enumerate}
			\item Se $f$ è misurabile allora $\abs{f}$ è misurabile. Basta considerare $\Phi(u,v) = \sqrt{u^2 + v^2}$.
			\item Se $f$ e $g$ sono misurabili allora $f+g$ e $fg$ sono misurabili. Basta considerare $\Phi(u,v) = u+v$ e $\Phi(u,v) = uv$. 
		\end{enumerate}
		Se le funzioni considerate sono a valori complessi, si effettua un passaggio preliminare sfruttando la misurabilità di parte reale e parte immaginaria.
	\end{oss}
	
	\begin{definition} [Funzione caratteristica]
		Sia $E\subseteq X$, si dice funzione caratteristica e si denota con $\rchi_E$ la funzione
		\begin{equation*}
			\rchi_E(x) = 
			\begin{cases}
				1 & x\in E \\
				0 & \textup{altrimenti}
			\end{cases}
		\end{equation*}
		Osserviamo che $\rchi_E$ è misurabile se e solo se $E$ è misurabile.
	\end{definition}
	
	\begin{theorem}
		Sia $X\neq \emptyset$ insieme e $\mathscr{F}$ una collezione di suoi sottoinsiemi, allora esiste la più piccola $\sigma$-algebra su $X$ contenente $\mathscr{F}$.
	\end{theorem}
	\begin{proof}
		Sia $\Sigma$ la famiglia di $\sigma$-algebre su $X$ contenenti $\mathscr{F}$. Questo insieme è non vuoto perché $\mathscr{P}(X)$ è una $\sigma$-algebra e contiene banalmente $\mathscr{F}$. Consideriamo dunque
		\begin{equation*}
			\mathscr{M} \defeq \bigcap_{S\in\Sigma}S
		\end{equation*}
		e mostriamo che è una $\sigma$-algebra.
		\begin{itemize}
			\item Dal momento che per ogni $S\in\Sigma$ si ha $X\in S$, otteniamo $X\in\mathscr{M}$;
			\item Se $E\in\mathscr{M}$ allora $\forall S\in \Sigma \, E\in S$ e quindi $\forall S\in \Sigma \, E^C\in S$, da cui $E^C \in \mathscr{M}$;
			\item Se $\suc{E}{n}{\N}\subseteq\mathscr{M}$, allora ogni $E_n$ è in tutti gli $S\in\Sigma$, e quindi l'unione degli $E_n$ è in tutti gli $S\in\Sigma$, da cui $\bigcup_{n\in\N}E_n\in \mathscr{M}$.
		\end{itemize}
	\end{proof}
	
	\begin{definition}[$\sigma$-algebra generata]
		Sia $X\neq \emptyset$ insieme e $\mathscr{F}$ una collezione di suoi sottoinsiemi, la più piccola $\sigma$-algebra in $X$ contenente $\mathscr{F}$ si dice $\sigma$-algebra generata da $\mathscr{F}$.
	\end{definition}
	
	\begin{definition}
		Sia $(X,\tau)$ uno spazio topologico , la $\sigma$-algebra generata da $\tau$ si dice $\sigma$-algebra di Borel, $(X,\mathscr{B})$, e i suoi elementi si dicono boreliani.
	\end{definition}
	
	\begin{nota}
		In $\R$ con la topologia naturale, tutti i boreliani sono misurabili secondo Lebesgue, ma esistono misurabili non boreliani per questioni di cardinalità. Tuttavia, per ogni insieme Lebesgue-misurabile $A$ esiste un boreliano $G$ tale che $A\setminus G$ ha misura nulla.
		
		Essendo i boreliani la $\sigma$-algebra generata dagli aperti, la famiglia dei boreliani contiene i chiusi, la famiglia di insiemi ottenuti dall'unione qualsiasi di chiusi (detti insiemi di tipo $F_\sigma$) e la famiglia di insiemi ottenuti dall'intersezione qualsiasi di aperti (detti insiemi di tipo $G_\delta$).
	\end{nota}
	
	\begin{prop}
		Siano $(X,\tau)$, $(Y,\gamma)$ spazi topologici, e consideriamo i boreliani $\mathscr{B}(X)$. Se una funzione $f: X\to Y$ è continua, allora $f$ è misurabile.
	\end{prop}
	\begin{proof}
		L'antimmagine di un aperto è un aperto, quindi un boreliano, quindi misurabile.
	\end{proof}
	
	\begin{prop}
		Siano $(X,\mathscr{M})$ spazio misurabile, $(Y,\tau)$ spazio topologico. Data una funzione $f:X \to Y$, si ha che:
		\begin{enumerate}
			\item La collezione di tutti i sottoinsiemi $\left\{ E\subseteq Y \,\middle\vert\, f^{-1}(E)\in\mathscr{M} \right\}$ è una $\sigma$-algebra in $Y$.
			\item Se $f$ è misurabile e $E\in\mathscr{B}(Y)$, si ha che $f^{-1}(E) \in \mathscr{M}$;
			\item Sia $Y=[-\infty,+\infty]$. Allora $f$ è misurabile se e solo se $\forall a\in [-\infty,+\infty)$ si ha $f^{-1}\left( [a,+\infty] \right) \in \mathscr{M}$.
		\end{enumerate}
	\end{prop}
	\begin{proof}
		Mostriamo la prima. Sia $\mathscr{U}=\left\{ E\subseteq Y \,\middle\vert\, f^{-1}(E)\in\mathscr{M} \right\}$ e mostriamo che verifica la definizione di $\sigma$-algebra:
		\begin{itemize}
			\item Dal momento che $f^{-1}(Y)=X\in\mathscr{M}$ si ha che $Y\in\mathscr{U}$.
			\item Sia $E\in\mathscr{U}$, da cui $f^{-1}(E)\in\mathscr{M}$. Allora anche $\left(f^{-1}(E)\right)^C\in\mathscr{M}$, quindi
			\begin{equation*}
				X\setminus f^{-1}(E) = f^{-1}(Y)\setminus f^{-1}(E) = f^{-1}(Y\setminus E) = f^{-1}(E^C) \,,
			\end{equation*}
			da cui si trova $E^C\in\mathscr{U}$
			\item Siano $\suc{E}{n}{\N}\subseteq \mathscr{U}$. Allora per ogni $n\in\N$ si ha $f^{-1}(E_n)\in\mathscr{M}$ e quindi l'unione di queste antimmagini, che è l'antimmagine dell'unione, è ancora in $\mathscr{M}$, da cui $\bigcup_{n\in\N}E_n\in\mathscr{U}$.
		\end{itemize}
		
		Mostriamo la seconda. La collezione $\mathscr{U}$ relativa alla funzione misurabile $f$ contiene per definizione tutti gli aperti di $\tau$. Dunque, passando alla $\sigma$-algebra generata, si ha che $\tau\subseteq\mathscr{U}\implies \mathscr{B}(Y)\subseteq\mathscr{U}$, da cui $f^{-1}(E)\in\mathscr{M}$ per ogni boreliano $E\subseteq Y$.
		
		Mostriamo la terza. L'implicazione ($\implies$) è banale, dal momento che $[a,+\infty]$ è un boreliano. Viceversa, se per ogni $a\in[-\infty,+\infty)$ sono in $\mathscr{M}$, ogni aperto di $Y$ si può scrivere come unione o intersezione di questi insiemi, da cui si ha $\tau\subseteq\mathscr{U}$ e quindi $f^{-1}(V)\in\mathscr{M}$ per ogni aperto $V$.
	\end{proof}
	
	
\end{document}