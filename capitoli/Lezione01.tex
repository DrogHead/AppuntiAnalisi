\documentclass[../main.tex]{subfiles}

\begin{document}
	
	\begin{definition}[Topologia]
		Sia $X$ un insieme non vuoto e sia $\tau$ una famiglia di sottoinsiemi di $X$. La coppia $(X,\tau)$ si dice ``spazio topologico'' e $\tau$ si dice ``topologia'' se:
		\begin{enumerate}
			\item $X\in\tau$ e $\varnothing\in\tau$;
			\item $\suc{A}{i}{I}\subseteq\tau \implies \bigcup_{i\in I} A_i \in \tau$;
			\item $\ennu{A}{n}\subseteq\tau \implies \bigcap_{i=1}^n A_i \in \tau$.
		\end{enumerate}
		Gli elementi di $\tau$ vengono detti ``aperti'', e i loro complementari vengono detti ``chiusi''.
	\end{definition}
	
	\begin{nota}
		Quando non è necessario specificare la topologia $\tau$, la coppia $(X,\tau)$ si indica solo con $X$.
	\end{nota}
	
	\begin{definition}[Funzione continua]
		Siano $(X,\tau)$ e $(Y,\sigma)$ spazi topologici. Una funzione $f : X\to Y$ si dice ``continua'' se $\forall V\in\sigma$ si ha $f^{-1}(V)\in\tau$. IN particolare, $f$ si dice continua nel punto $x_0\in X$ se per ogni intorno $V$ di $f(x_0)$ esiste un intorno $U$ di $x_0$ tale che $f(U)\subseteq V$.
	\end{definition}
	
	\begin{definition}[$\sigma$--algebra]
		Sia $X$ un insieme non vuoto e sia $\mathscr{M}$ una famiglia di sottoinsiemi di $X$. $\mathscr{M}$ si dice ``$\sigma$--algebra'' se:
		\begin{enumerate}
			\item $X\in\mathscr{M}$;
			\item $A\in\mathscr{M} \implies A^C\in\mathscr{M}$;
			\item $\suc{A}{i}{\N}\in\mathscr{M} \implies \bigcup_{i\in \N} A_i \in \mathscr{M}$.
		\end{enumerate}
		La coppia $(X,\mathscr{M})$ è detta ``spazio misurabile'', e gli elementi di $\mathscr{M}$ sono detti ``misurabili''.
	\end{definition}
	
	\begin{nota}
		È immediato dedurre che le $\sigma$--algebre contengono le intersezioni numerabili dei propri elementi dato che
		\begin{equation*}
			\bigcap_{i\in I} A_i = \big(\bigcup_{i\in I} A_i^C \big)^C \,.
		\end{equation*}
	\end{nota}
	
	\begin{definition}[Funzione misurabile]
		Dati uno spazio misurabile $(X,\mathscr{M})$ e uno spazio topologico $(Y,\sigma)$, una funzione $f : X\to Y$ si dice ``misurabile'' se $\forall V\in \sigma$ si ha $f^{-1}(V)\in\mathscr{M}$.
	\end{definition}
	
	\begin{definition}[Spazio metrico]
		Sia $X$ un insieme non vuoto. La funzione $d : X\times X \to \R^+$ si dice ``distanza'' o ``metrica'' se:
		\begin{enumerate}
			\item $\forall x,y\in X \quad d(x,y)=d(y,x)$;
			\item $d(x,y)=0 \iff x=y$;
			\item $\forall x,y,z\in X \quad d(x,y)\leq d(x,z)+d(z,y)$.
		\end{enumerate}
		La coppia $(X,d)$ è detta ``spazio metrico''.
	\end{definition}
	
	\begin{nota}
		Dato uno spazio metrico $(X,d)$, consideriamo gli insiemi della forma
		\begin{equation*}
			B_\eps (x_0) = \{ x\in X \vert c d(x,x_0)<\eps \}
		\end{equation*}
		per ogni $\eps>0$ e $x_0\in X$, detti palle aperte di raggio $\eps$ centrate in $x_0$. Si dice ``topologia indotta'' su $X$ da $d$ e si indica con $\tau_d$ la famiglia di unioni qualsiasi e intersezioni finite di palle.
	\end{nota}
	
	\begin{prop}
		Una $\sigma$-algebra è finita o più che numerabile.
	\end{prop}
	\begin{proof}
		Sia $(X,\mathscr{M})$ uno spazio misurabile, e per assurdo sia $\mathscr{M}=\suc{A}{i}{\N}$ con $A_i\neq A_j$. Per ogni $x\in X$, definiamo l'insieme
		\begin{equation*}
			B_x = \bigcap_{x\in A\in \mathscr{M}} A \,,
		\end{equation*}
		ossia l'intersezione dei misurabili contenenti $x$. Presi $x,y\in X$ e considerati $B_x$ e $B_y$, sono possibili i seguenti casi:		
		\begin{enumerate}
			\item $B_x \bigcap B_y = \emptyset$;
			\item $B_x = B_y$;
			\item $B_x\bigcap B_y=C$.
		\end{enumerate}
		Osservato che $B_x$ e $B_y$ sono i più piccoli insiemi misurabili contenenti rispettivamente $x$ e $y$, si conclude immediatamente che il terzo caso si riconduce al secondo. Dunque, la famiglia $\suc{B}{x}{X}$ forma una partizione di $X$. Inoltre, se $A\in\mathscr{M}$, si ha
		\begin{equation*}
			A = \bigcup_{x\in A} B_x \,,
		\end{equation*}
		e dunque si hanno due possibilità: se $B_x=B_y$ i $B_i$ sono finiti di numero allora esisterà un numero finito di possibili combinazioni di $B_x$ per formare gli elementi di $\mathscr{M}$, che quindi è finito, il che è assurdo. L'altra possibilità è che i $B_x$ siano tutti diversi tra loro, dunque si avrà un numero più che numerabile di possibili ricoprimenti, quindi assurdo.
		
	\end{proof}
	
	% DA QUI NON HO POTUTO STAMPARE NON SO COME VIENE
	
	\begin{prop}[Composizione funzioni continue su spazi topologici]
		Siano $ (X,\tau _X), (Y,\tau _Y)$ e $ (Z,\tau_Z)$ spazi topologici, e siano $f: X\to Y$ e $g:Y\to Z$ funzioni, se $f$ e $g$ sono continue allora $g \circ f$ è continua
	\end{prop}
	\begin{proof}
		Sia $U\in \tau_Z$, si vuole mostrare che $ (g \circ f)^{-1} (U) \in \tau_X$. Data la continuità di $g$ si ha che $g^{-1}(U) \in \tau_Y$, dunque per la continuità di $f$ si ha $f^{-1}(g^{-1}(U)) = (g \circ f)^{-1} (U) \in \tau_X$.
	\end{proof}
	
	\begin{prop}[Composizione di funzioni su spazi topologici e misurabili]
		Siano $ (X,\tau_X, \mathscr{M}_X)$, $(Y,\tau_Y, \mathscr{M}_Y)$, $(Z,\tau_Z, \mathscr{M}_Z) $ spazi topologici misurabili, siano le funzioni  $f: X\to Y$ misurabile e $g:Y\to Z$ continua, allora $g \circ f$ è misurabile.
	\end{prop}
	\begin{proof}
		Sia $U\in \tau_Z$, si vuole mostrare  che $ (g \circ f)^(-1) (U) \in \mathscr{M}_X$. Data la continuità di $g$ si ha che $g^(-1)(U) \in \tau_Y$, dunque per la misurabilità di $f$ si ha $f^(-1)(g^(-1)(U)) = (g \circ f)^(-1) (U) \in \mathscr{M}_X$.
	\end{proof}
	
	\begin{prop}[Composizione di funzioni misurabili]
		Siano $u,v: X \to \mathbb{R}$ funzioni reali misurabili, sia $\Phi: \mathbb{R}^2\to Y$ continua; allora $\Phi(u(x),v(x))$ è misurabile.
	\end{prop}
	\begin{proof}
		Per semplificare le notazioni si indica con $f: x\in X \to (u(x),v(x))\in \mathbb{R}^2$, dunque la tesi è mostrare la misurabilità di $\Phi ( f(x))$.\\
		Si considera un sottoinsieme aperto di Y, $V \subseteq Y$ , $\Phi^(-1) (V) \coloneq W$ è un aperto di $\mathbb{R}^2$ data $\Phi$ continua, dunque è un unione numerabile di rettangoli aperti: $W= \bigcup_{i\in I} R_i $. Per mostrare che  $\Phi(u(x),v(x))$ è misurabile bisogna mostrare che l'antimagine di un aperto di Y è misurabile: $ (\Phi \circ f)^{-1} (V)= f^{-1}(\Phi^{-1}(V))=f^{-1}(W)=f^{-1}(\bigcup_{i\in I} R_i )=\bigcup_{i\in I} f^{-1}(R_i)\in \mathscr{M}$ dato che si tratta di antimmagini di intervalli aperti tramite $u$ e $v$ che sono misurabili per ipotesi.
	\end{proof}
	
	\begin{corol}[Corollario-osservazione]
		La funzione $f:\mathbb{R}\in \mathbb{C} $ complessa,  è misurabile se e soltanto se la sua parte immaginaria e quella reale sono misurabili.
	\end{corol}
	\begin{proof}
		Si dimostra prima da destra verso sinistra.\\
		Si tratta di un caso particolare della proposizione precedente: infatti considerando le funzioni  $ u,v: \mathbb{R}\to \mathbb{R}$, si esprime $f: x\in \mathbb{R} \to u(x)+iv(x)\in \mathbb{C}.$. Ovviamente f è una funzione continua, dunque per la proposizione precedente ogni volta che la parte reale $u$ e la parte immaginaria $v$ sono misurabili, f è misurabile
		Per mostrare l'altra implicazione basta osservare che data $f$ misurabile , se la compongo con la funzione continua che individua la sua parte reale $u$ o con quella che individua la sua parte immaginaria $v$, allora il risultato, che è la parte reale o immaginaria, risulta misurabile per una proposizione precedente.
		
	\end{proof}
	
	\begin{oss}
		\begin{enumerate}
			\item Il ragionamento fatto per la funzione complessa vale per la funzione valore assoluto: $|f|= \sqrt{u^2+v^2}$, il modulo è una funzione continua e se la compongo con f misurabile si ha che il modulo di f è misurabile.
			\item La somma e il prodotto di $f,g$ funzioni complesse è misurabile, si mostra osservando che se sono misurabili lo sono anche le loro parti reali e immaginarie, ed effettuando la somma e il prodotto tra $f$ e $g$ si sommano e moltiplicano le parti immaginarie tra loro e lo stesso per quelle reali, ciò produce funzioni misurabili .
		\end{enumerate}
	\end{oss}
	
	\begin{definition} [Funzione caratteristica]
		Sia $E\subseteq X$, si dice funzione caratteristica e si denota con $\rchi_E$ la funzione
		\begin{equation*}
			\rchi_E(x) = 
			\begin{cases}
				1 & x\in E \\
				0 & \textup{altrimenti}
			\end{cases}
		\end{equation*}
	\end{definition}
	
	\begin{oss}
		Se l'insieme $E$ è misurabile, allora la sua funzione caratteristica è misurabile.
	\end{oss}
	
	\maketitle
	$\sigma$ - Algebra generata
	\begin{theorem}
		Sia $X\neq \emptyset$ insieme e $\mathscr{F}$ una collezione di suoi sottoinsiemi, allora esiste la più piccola $\sigma$-algebra in X  contente  $\mathscr{F}$.
	\end{theorem}
	\begin{proof}
		Sia U la famiglia di $\sigma$-algebre contenenti  $\mathscr{F}$, $U\neq \emptyset$ dato che la $\sigma$-algebra banale (tutti i sottoinsiemi di X) appartiene a U. Sia $\mathscr{M}^*=\bigcap_{\mathscr{M}\in U}\mathscr{M}$, l'obiettivo è mostrare che è una $\sigma$-algebra.
		\begin{itemize}
			\item $\mathscr{M}^*$ è chiusa per completamento? Sia $A \in \mathscr{M} \implies A\in \mathscr{M}^* \forall \mathscr{M} \in U \implies A^C \in \mathscr{M} \forall \mathscr{M} \in U \implies A^C \in \bigcap_{\mathscr{M}\in U} \mathscr{M}= \mathscr{M}^*$.
			\item $\mathscr{M}$ è chiusa per uninione numerabile? Sia $\{A_i \}_{i\in \mathbb{N}} \in \mathscr{M}^* \implies A_i \in \mathscr{M} \forall \mathscr{M}\in U \forall i\in \mathbb{N} \implies \bigcup_{i\in\mathbb{N}}A_i \in \bigcap_{\mathscr{M}\in U } \mathscr{M}= \mathscr{M}^*$
			\item $X \in \mathscr{M}\in U$ ? Ovvio.
		\end{itemize}
	\end{proof}
	
	La $\sigma$- algebra la cui esistenza è appena stata provata prende il nome di $\sigma$-algebra generata.
	
	\begin{definition}[ $\sigma$-algebra generata]
		Sia $X\neq \emptyset$ insieme e $\mathscr{F}$ una collezione di suoi sottoinsiemi, la più piccola $\sigma$-algebra in X  contente  $\mathscr{F}$ si dice $\sigma$-algebra generata da $\mathscr{F}$.
	\end{definition}
	
	\begin{definition}
		Sia $(X,\tau)$ uno spazio topologico , la $\sigma$-algebra generata da $\tau$ si dice $\sigma$-algebra di Borel, $(X,\mathscr{B})$, e i suoi elementi si dicono boreliani.
	\end{definition}
	
	\begin{nota}
		In $\R$ con la topologia naturale, tutti i boreliani sono misurabili secondo Lebesgue, ma esistono misurabili non boreliani per questioni di cardinalità. Tuttavia, per ogni insieme Lebesgue--misurabile $A$ esiste un boreliano $G$ tale che $A\setminus G$ ha misura nulla.\\
		Essendo i boreliani la $\sigma$--algebra generata dagli aperti, la famiglia dei boreliani contiene i chiusi, la famiglia di insiemi ottenuti dall'unione qualsiasi di chiusi (detti insiemi di tipo $F_\sigma$) e la famiglia di insiemi ottenuti dall'intersezione qualsiasi di aperti (detti insiemi di tipo $G_\delta$).
	\end{nota}
	
	\begin{prop}
		Sia $(X,\mathscr{B})$ uno spazio misurabile, con $\mathscr{B}$ generato dalla topologia $\tau$ e sia $(Y,\tau)$ uno spazio topologico, sia $f:X \to Y$ continua, allora $f$ è misurabile.
	\end{prop}
	\begin{proof}
		L'antimmagine di un aperto è un aperto, quindi un boreliano, quindi misurabile.
	\end{proof}
	
	\begin{prop}
		Siano $(X,\mathscr{M})$ spazio misurabile, $(Y,\tau)$ spazio topologico e $f:X \to Y$ funzione, si ha che:
		\begin{enumerate}
			\item La collezione di tutti i sottoinsiemi $E\subseteq Y | f^{-1} (E) \in \mathscr{M}$ è una $\sigma$-algebra in $Y$.
			\item Se $f$ è misurabile e $E$ è di Borel in $Y$ , si ha che $f^{-1} \in \mathscr{M}$;
			\item Sia $Y=[-\infty,+\infty]$, $f$ è misurabile $ \forall a\in \mathbb{R} f^{-1}([a,+\infty])\in \mathscr{M}$;
		\end{enumerate}
	\end{prop}
	\begin{proof}
		\begin{itemize}
			\item Sia $U$ la collezione dei sottoinsiemi, si verifica dunque  la definizione di $\sigma$-algebra: $Y\in U$ dato che $f^{-1}(Y)=X$\\
			Sia $E\subseteq Y $ , se $E\in U$ l'obiettivo è mostrare che $f^{-1}(E^C) \in U$: dato che $f^{-1}(E)\in \mathscr{M} \implies X/ f^{-1}(E)=f^{-1}(Y/E)=E^XC\in \mathscr{M}$.\\
			Sia $E_i \in U \forall i \in \mathbb{N}$ l'obiettivo è mostrare che $\bigcup_{i\in \mathbb{N}}E_i \in U$. ma dato che  $ f^{-1}(E_i)\in \mathscr{M} \forall i \in \mathbb{N} \implies \bigcup_{i\in \mathbb{N}}E_i \in \mathscr{M} \implies f^{-1}( \bigcup_{i\in \mathbb{N}}E_i )\in \mathscr{M}. $\\
			
			\item SI consideri U la collezione definita nel punto precendete, la misurabilità di $f$ implica che $U$ contiene tutti gli aperti di $\tau$, ma dato che $U$ è anche una $\sigma$-algebra per il punto precedente, si ha che deve necessariamente contenere la $\sigma$-algebra di Borel. Quindi dato un boreliano, questo verifica la tesi in quento appartente ad $U$.
			
			\item Sia $U$ una collezione dei sottoinsiemi $ E \subseteq [-\infty,\infty]$ tali che $f^{-1}(E)\in \mathscr{M} $. Sia $a\in \mathbb{R}$, considero $\{a-\frac{1}{n} \}_n \to a $ successione. Dato che $[-\infty,a]=\bigcup_n]-\infty,a_n] = \bigcup_n [a_n, \infty[^C$ si ha	$] a-\frac{1}{n},\infty[\in U \ \forall n$. Dunque si può dedurre che questi intervalli aperti possono generare la $\sigma$-algebra dei boreliani: per ogni intervallo $(\alpha, \beta) = ]-\infty,\beta] \cap ]\alpha, \infty[$.
			%% HO COSì DIMOSTRATO DA SX VERSO DX, IL VICEVERSA è OVVIO , LUI HA PERSINO ENU]NCIATO DIVERSO
		\end{itemize}
	\end{proof}
	
	
\end{document}