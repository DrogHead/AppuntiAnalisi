\documentclass[../main.tex]{subfiles}

\begin{document}
	
	\begin{definition}[Insieme di Cantor]
		L'insieme di Cantor $C$ si trova come limite di una successione decrescente di insiemi di misura finita, contenuti in $[0,1]$. Definiamo $C_0=[0,1]$, lo dividiamo in terzi e rimuoviamo il terzo centrale, ottenendo
		\begin{equation*}
			C_1 = \left[0,\frac13\right] \cup \left[\frac23,1\right] = I_1^1 \cup I_1^2 \,.
		\end{equation*}
		Dividiamo entrambi gli intervalli $I_1^1$ e $I_1^2$ in terzi e rimuoviamo il terzo centrale, ottenendo
		\begin{equation*}
			C_2 = \left[0,\frac19\right] \cup \left[\frac29,\frac13\right] \cup \left[\frac23,\frac79\right] \cup \left[\frac89,1\right] = I_2^1 \cup I_2^2 \cup I_2^3 \cup I_2^4 \,.
		\end{equation*}
		Al passo $n$-esimo l'insieme $C_n$ è unione disgiunta di $I_n^1,\dots,I_n^{2^n}$ intervalli di ampiezza $3^{-n}$, prendiamo ognuno, lo dividiamo in terzi e rimuoviamo il terzo centrale, ottenendo $C_{n+1}$ come unione disgiunta di $I_{n+1}^1,\dots,I_{n+1}^{2^{n+1}}$ intervalli di ampiezza $3^{-(n+1)}$. Ovviamente $C_{n+1}\subseteq C_n$. L'insieme di Cantor $\mathscr{C}$ è il limite dei $C_n$.
	\end{definition}
	
	\begin{prop}
		L'insieme di Cantor è un insieme misurabile, con la potenza del continuo, e ha misura nulla.
	\end{prop}
	\begin{proof}
		Essendo intersezione numerabile di misurabili, anche $\mathscr{C}$ è misurabile, e la sua misura è data da
		\begin{equation*}
			m(\mathscr{C}) = \lim_{n\to\infty} m(C_n) = \lim_{n\to\infty} \frac{2^n}{3^n} = 0 \,.
		\end{equation*}
		Per mostrare che $\mathscr{C}$ ha la potenza del continuo, scriviamo l'espansione decimale dei reali in $[0,1]$ in base 3, in cui le cifre sono 0,1,2. Gli elementi di $\mathscr{C}$ sono tutti e soli quelli che in base 3 si scrivono con le cifre 0 e 2, ma comunque hanno la potenza del continuo.
	\end{proof}
	
	\begin{definition}[Insieme perfetto]
		Un insieme perfetto è un insieme chiuso senza punti isolati. L'insieme di Cantor è un esempio particolare di insieme perfetto, in quanto è totalmente sconnesso e ha misura nulla.
	\end{definition}
	
	\begin{definition}[Funzione di Cantor-Vitali]
		La funzione di Cantor-Vitali si ottiene come limite di funzioni continue e crescenti definite tramite l'insieme di Cantor. Partiamo da $f_0(x)=x$, che è il raccordo lineare tra $(0,0)$ e $(1,1)$. Costruiamo $f_1$ usando l'insieme $C_1$:
		\begin{itemize}
			\item Su $[0,1/3]$, $f_1(x)$ è il raccordo lineare tra i punti $(0,0)$ e $(1/3,1/2)$
			\item Su $[1/3,2/3]$, $f_1(x)$ è la funzione costantemente uguale a $1/2$
			\item Su $[2/3,1]$, $f_1(x)$ è il raccordo lineare tra i punti $(2/3,1/2)$ e $(1,1)$
		\end{itemize}
		La funzione $f_n$ è quindi il raccordo lineare sui $2^n$ intervalli disgiunti di $C_n$, ed è costante sui rimanenti intervalli, in modo tale che $f_n$ sia continua. Le funzioni $f_n$ sono lipschitziane con costante $L=\left(\frac32\right)^n$. La funzione di Cantor-Vitali è definita come
		\begin{equation*}
			f(x) = f_0(x) + \sum_{n=0}^\infty (f_{n+1}(x)-f_n(x)) \,,
		\end{equation*}
		che è una serie telescopica e quindi $f=\lim_N f_N$, che converge totalmente perché la differenza tra $f_{N+1}$ e $f_N$ è al massimo $2^{-N}$.
	\end{definition}
	
	\begin{oss}
		La funzione di Cantor è continua essendo limite di funzioni continue, ma non è invertibile in quanto non iniettiva, essendo costante su un'infinità numerabile di segmenti.
		
		La funzione di Cantor è quasi ovunque derivabile, e la sua derivata è quasi ovunque nulla. Non vale il teorema fondamentale del calcolo integrale perché $f(1)-f(0)=1$ ma l'integrale di $f'$ è nullo.
		
		L'immagine mediante $f$ dell'insieme di Cantor è un sottoinsieme di $[0,1]$ con misura $m(f(\mathscr{C}))=1$.
		
		La funzione di Cantor è $\alpha$-h\"{o}lderiana con $\alpha=\log_3 2$.
	\end{oss}
	
	\begin{oss}
		Sia $f$ la funzione di Cantor e sia $g(x)=x+f(x)$, che è strettamente crescente e quindi iniettiva. Allora $g([0,1])=[0,2]$. La funzione $g^{-1}$ esiste ed è continua, e quindi mappa boreliani in borelinai, da cui segue che $m(g(C))=1$.
	\end{oss}
	
\end{document}








