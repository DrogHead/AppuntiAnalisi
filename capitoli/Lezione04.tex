\documentclass[../main.tex]{subfiles}

\begin{document}
	
	\begin{definition}
		Lo spazio di misura $(X,\mathscr{M},\mu)$ si dice completo (o $\mu$ si dice completa) se tutti i sottoinsiemi degli insiemi di misura nulla hanno a loro volta misura nulla.
	\end{definition}
	
	\begin{theorem}
		Per ogni spazio di misura $(X,\mathscr{M},\mu)$ esiste uno spazio di misura completo $(X,\mathscr{M}^*, \mu^*)$, tale che $\mathscr{M}\subseteq\mathscr{M}^*$ e che $\mu^*$ ristretta a $\mathscr{M}$ coincide con $\mu$.
	\end{theorem}
	\begin{proof}
		La $\sigma$-algebra $\mathscr{M}^*$ si definisce come
		\begin{equation*}
			E\in\mathscr{M}^* \iff \exists A,B\in\mathscr{M} : A\subseteq E\subseteq B \,, \mu(B\setminus A) = 0 \,,
		\end{equation*}
		e si mostra che $\mathscr{M}^*$ rispetta la definizione. Inoltre, definita $\mu^*(E) = \mu(A) = \mu(B)$, si fa vedere che $\mu^*$ è una misura ed completa.
	\end{proof}
	
	\begin{lemma}
		Sia $f :X\to\R$ una funzione misurabile non negativa e sia $E\in\mathscr{M}$ tale che $\int_E f\dmu = 0$. Allora $f$ è quasi ovunque nulla in $E$.
	\end{lemma}
	\begin{proof}
		Sia $E_n = \left\{x\in E \,\middle\vert\, f>\frac1n \right\}$ per ogni $n\in\N$ e definiamo $E = \bigcup_n E_n = \{f\ne 0\}$. Allora per ogni $n\in\N$ si ha
		\begin{equation*}
			0\leq \frac1n \mu(E_n) \leq \int_{E_n}f\dmu \leq \int_E f\dmu = 0 \,,
		\end{equation*}
		da cui $\mu(E_n)=0$ per ogni $n\in\N$ e quindi $\mu(E)=0$, che è la tesi.
	\end{proof}
	
	\begin{corol}
		Il lemma che abbiamo appena dimostrato vale anche per le funzioni $L^1$.
	\end{corol}
	
	\begin{prop}
		Sia $\suc{E}{n}{\N}\subseteq\mathscr{M}$ tale che $\sum_n \mu(E_n)<+\infty$. Allora ogni $x\in X$ appartiene al più a un numero finito di $E_n$.
	\end{prop}
	\begin{proof}
		Consideriamo le funzioni misurabili
		\begin{equation*}
			f_N = \sum_{n=1}^N \rchi_{E_n} \,\qquad f = \sum_{n=1}^\infty \rchi_{E_n}
		\end{equation*}
		e notiamo che $f(x)=+\infty$ se e soltanto se $x$ si trova in un numero infinito di insiemi $E_n$. Per il teorema di convergenza monotona, si trova che
		\begin{equation*}
			\lim_n \int_X f_n\dmu = \int_X f\dmu = \sum_{n=1}^\infty \mu(E_n) < +\infty
		\end{equation*}
		e quindi concludiamo che $f$ è quasi ovunque finita, ossia che quasi ogni $x\in X$ è in un numero finito di $E_n$.
	\end{proof}
	
	\begin{theorem}[Teorema di derivazione sotto segno di integrale]
		Sia $(X,\mathscr{M},\mu)$ uno spazio di misura e consideriamo una funzione $f :X\times(a,b)\to\R$ tale che
		\begin{enumerate}
			\item $f(\cdot,y) : X\to \R$ è sommabile per ogni $y\in(a,b)$;
			\item $f(x,\cdot) : (a,b)\to \R$ è derivabile per quasi ogni $x\in X$;
			\item $\exists g\in L^1(X,\mu)$ tale che $\abs{\partial_y f(x,y)}\leq g(x)$ per quasi ogni $x\in X$.
		\end{enumerate}
		Allora si ha che $y \mapsto \int_X f(x,y) \dmu_x$ è derivabile, e che
		\begin{equation*}
			\dfrac{\partial}{\partial y}\int_X (x,y)\dmu_x = \int_X \partial_yf(x,y) \dmu_x \,.
		\end{equation*}
	\end{theorem}
	\begin{proof}
		Definiamo la funzione ausiliaria
		\begin{equation*}
			F_h(x,y) = \frac{ f(x,y+h)-f(x,y) }{h} \,,
		\end{equation*}
		e osserviamo che il limite per $h\to 0$ di $F_h$ è proprio la derivata $\partial_y f$. Dal teorema di Lagrange, $\exists\theta\in\R$ dipendente da $x$ e da $y$, con $\abs{\theta}<h$ e tale che
		\begin{equation*}
			F_h(x,y) = \partial_y f(x,y+\theta) \,.
		\end{equation*}
		Usando il teorema di convergenza dominata (con parametro continuo, ma vale per ogni successione e quindi si può passare per il teorema ponte) e troviamo che
		\begin{equation*}
			\dfrac{\partial}{\partial y}\int_X F_h(x,y)\dmu_x = \lim_{h\to 0}\int_X F_h(x,y)\dmu_x = \int_X \lim_{h\to0} F_h(x,y) \dmu_x = \int_X \partial_y(x,y) \dmu_x \,.
		\end{equation*}
	\end{proof}
	
\end{document}












