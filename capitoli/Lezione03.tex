\documentclass[../main.tex]{subfiles}

\begin{document}
	
	\begin{corol}[Serie di funzioni]
		Dato uno spazio di misura $(X,\mathscr{M},\mu)$ e una successione di funzioni $\suc{f}{n}{\N}$ misurabili non negative, per il teorema di convergenza monotona si ha
		\begin{equation*}
			\sum_{n=1}^\infty \int_X f_n\dmu = \int_X \sum_{n=1}^\infty f_n \dmu \,.
		\end{equation*}
	\end{corol}
	
	\begin{theorem}[Lemma di Fatou]
		Dato uno spazio di misura $(X,\mathscr{M},\mu)$ e una successione di funzioni $\suc{f}{n}{\N}$ misurabili non negative, si ha
		\begin{equation*}
			\int_X \liminf_n f_n \dmu \leq \liminf_n \int_X f_n\dmu
		\end{equation*}
	\end{theorem}
	\begin{proof}
		Dalla definizione di $\liminf$, consideriamo la successione $g_n = \inf_{k\geq n}f_k$, per cui $g_n \leq f_k$ per ogni $k\geq n$ e il limite per $n\to\infty$ delle $g_n$ è $\liminf_n f_n$. Ovviamente anche $\suc{g}{n}{\N}$ è una successione di funzioni misurabili non negative, ed è crescente in $n$. Usando il teorema di convergenza monotona si ha
		\begin{equation*}
			\lim_n \int_X g_n \dmu = \int_X \lim_n g_n d\mu \,.
		\end{equation*}
		Usando la monotonia dell'integrale si trova
		\begin{equation*}
			\int_X g_n \dmu \leq \inf_{k\geq n} \int_X f_k \dmu \,,
		\end{equation*}
		e quindi passando al limite per $n\to\infty$ si ha
		\begin{equation*}
			\int_X \liminf_n f_n \dmu = \lim_n \int_X g_n \dmu \leq \lim_n \inf_{k\geq n} \int_X f_k \dmu = \liminf_n \int_X f_n \dmu \,.
		\end{equation*}
	\end{proof}
	
	\begin{nota}
		Il teorema di convergenza monotona vale anche nel caso in cui $\suc{f}{n}{\N}$ è una successione di funzioni misurabili non negativa e decrescente in $n$, aggiungendo l'ipotesi che una delle $f_k$ ha integrale finito.
	\end{nota}
	
	\begin{definition}[Integrale di funzione misurabile non negativa]
		Consideriamo uno spazio di misura $(X,\mathscr{M},\mu)$ e una funzione misurabile $f :X\to \R$. Si definisce l'integrale di $f$ su $X$ rispetto a $\mu$ come
		\begin{equation*}
			\int_X f\dmu \defeq \int_X f^+\dmu - \int_X f^-\dmu \,,
		\end{equation*}
		mentre se $f:X\to\C$ si definisce l'integrale come
		\begin{equation*}
			\int_X f\dmu \defeq \int_X \Re(f)\dmu + i\int_X \Im(f)\dmu \,.
		\end{equation*}
	\end{definition}
	
	\begin{definition}[Funzione sommabile]
		Si dice che $f:X\to\C$ è una funzione sommabile se $\int_X\abs{f}\dmu < +\infty$. L'insieme delle funzioni sommabili è denotato con $L^1(X,\mu)$, talvolta omettendo $X$ e $\mu$. L'insieme $L^1(X,\mu)$ è uno spazio vettoriale con le operazioni di somma e prodotto per un numero reale o complesso.
	\end{definition}
	
	\begin{prop}[Disuguaglianza triangolare]
		Sia $E\in\mathscr{M}$ e $f$ una funzione misurabile. Allora $\abs*{\int_E f\dmu}\leq\int_E\abs{f}\dmu$.
	\end{prop}
	\begin{proof}
		Se $f$ non è sommabile la disuguaglianza è banale. Se $f:X\to\R$ si trova immediatamente che
		\begin{equation*}
			\abs*{\int_E f\dmu} \leq \abs*{\int_E f^+\dmu} + \abs*{\int_E f^-\dmu} = \int_E f^++f^- \dmu = \int_E\abs{f}\dmu \,.
		\end{equation*}
		Se $f:X\to\C$, invece, detto
		\begin{equation*}
			\alpha = \frac{ \overline{ \int_E f\dmu } }{ \abs*{ \int_E f\dmu } } \,,
		\end{equation*}
		che ha modulo $1$. Allora si ha
		\begin{equation*}
			\abs*{\int_E f\dmu} = \alpha \int_E f\dmu = \int_E \alpha f\dmu = \int_E \Re(\alpha f)\dmu \leq \int_E \abs{f}\dmu \,.
		\end{equation*}
	\end{proof}
	
	\begin{theorem}[Teorema di convergenza dominata di Lebesgue]
		Sia $\suc{f}{n}{\N}$ una successione di funzioni misurabili su $X$, convergente puntualmente a $f$, e sia $g\in L^1(X,\mu)$ tale che per ogni $n\in\N$ si abbia $\abs{f_n}\leq g$ in $X$. Allora si ha
		\begin{equation*}
			\lim_n \int_X \abs{f_n-f}\dmu = 0 \,.
		\end{equation*}
	\end{theorem}
	\begin{proof}
		Dalla condizione su $g$ si trova che $-g \leq f_n \leq g$ per ogni $n\in\N$, da cui per permanenza del segno si trova anche $-g\leq f\leq g$. Usiamo il lemma di Fatou su $f_n+g\geq 0$ e troviamo
		\begin{align*}
			\int_X \liminf_n(f_n+g) \dmu &\leq \liminf_n \int_X (f_n+g)\dmu \\
			\implies \int_X f+g\dmu & \leq \liminf_n \left( \int_X f_n\dmu \right) + \int_X g\dmu \\
			\implies \int_X f\dmu &\leq \liminf_n \int_X f_n\dmu \,.
		\end{align*}
		Usiamo il lemma di Fatou su $g-f_n\geq 0$ e troviamo, ricordando che $\liminf_n -a_n = -\limsup_n a_n$, che
		\begin{align*}
			\int_X \liminf_n (g-f_n)\dmu &\leq \liminf_n \int_X (g-f_n)\dmu \\
			\implies \int_X g-f \dmu &\leq \int_X g\dmu + \liminf_n \left(-\int_X f_n\dmu\right) \\
			\implies \int_X f\dmu &\geq \limsup_n \int_X f_n \dmu \,.
		\end{align*}
		Unendo la due disuguaglianze si trova che il limite della successione $\int_Xf_n\dmu$ esiste e vale proprio $\int_X f\dmu$. La tesi si trova usando il teorema per la funzione $\varphi_n=\abs{f_n-f}$ che tende a $0$ e che è limitata da $2g\in L^1(X,\mu)$.
	\end{proof}
	
\end{document}












