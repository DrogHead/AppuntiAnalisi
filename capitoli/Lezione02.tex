\documentclass[../main.tex]{subfiles}

\begin{document}
	% lezione del 18/09
	\begin{oss}
		Considerando una successione $\{a_n\}_{n\in \N}\subset \R $ si ha  che $lim\ sup a_n = lim_n \ sup_{k\geq n} a_k = inf_n sup_{k\geq n} a_k = inf_n g_n$ fissando la successione decrescente $ g_n \coloneq sup_{k\geq n} a_k$.\\
		Analogamente $ lim \ inf_n \ a_n= lim_n inf_{k \geq n} a_n = sup inf_{k \geq n} a_n = sup g_n $ fissando la successione crescente $g_n \coloneq inf_{k \geq n} a_n$.
	\end{oss}
	
	\begin{prop}[Misurabilità di lim inf e lim sup]
		Sia $(X,\mathscr{M}) $ spazio misurabile e	$\{f_n\}_{n\in \N}$ un successione di funzioni misurabili $f_n(x): X \to [-\infty. \infty]$. Allora $lim \ sup_n f_n(x)$ e  $lim \ inf_n f_n(x)$ sono misurabili.
	\end{prop}
	\begin{proof}
		Si dimostra prima per il lim sup , l'idea è che sfruttando l'osservazione preliminare si ha che $lim_n sup_{k \geq n} f_n(x) = inf_n \ sup _{k\geq n} f_k(x) $, dunque se si dimostra che l'inf e il sup di successioni misurabili sono misurabili si ottiene l'asserto, in particolare  dato il punto 3 dell'ultima porposizione della lezione precedente, bisogna mostrare che l'antimmagine di $(a, \infty)$ è misurabile . \\
		Prima di tutto biogna verificare che $ z(x ) \coloneq sup_n f_n(x)$  è misurabile, quindi  che $z^{-1}([a, \infty]) \in \mathscr{M}$ : $ z^{-1} ([a,\infty])= z^{-1}(\{  x \in (a,\infty) \})= \{ x\ in \R | z(x) \in (a,\infty)\}= \bigcup_n f^{-1}_n ((a,\infty))$ ma essendo la successione misurabile per ipotesi si ha che è misurabile l'antimmagine di $f$, e l'unione di misurabili è misurabile. \\
		Per quanto riguarda l'inf, analogamnete a come fatto prima bisogna verificare che, fissata $z(x) \coloneq inf_n \ f(_n (x)) $, si ha $ z^{-1}((a,\infty)) \in \mathscr{M}$: con gli stessi passaggi di prima si ottiene $z^{-1}((a,\infty))=  \{ x \in \R | z(x) \in (a, \infty)\}=  \bigcup_n f^{-1}_n ((a,\infty))$  che è misurabile. \\
		Così si ottiene l'asserto $lim \ sup_n f_n(x)$ .\\
		Per quanto riguarda il $lim \ inf_n f_n(x)$, data l'osservazione precedente si ha che  $lim \ inf_n f_n(x)= sup_n \ inf_{k\geq n} f_n(x)$ , ma è appena stato dimostrato che l'inf e il sup di successioni misurabili è misurabile, dunque l'asserto è immedato.
	\end{proof}
	
	\begin{corol}
		Sia $(X,\mathscr{M}) $ spazio misurabile e	$\{f_n\}_{n\in \N}$ un successione di funzioni misurabili $f_n(x): X \to [-\infty. \infty]$, allora se esiste il limite della successione è misurabile.
	\end{corol}
	\begin{proof}
		Dato che esiste il limite questo è uguale al lim inf e al lim sup , che sono misurabili per la proposizione precedente.
	\end{proof}
	
	\begin{exercise}
		Sia $\{f_n(x)\}_{n\in \N}$ una successione di funzioni misurabili, sia $A \subseteq X$  l'insieme di convergenza $\implies \ A \in \mathscr{M}$.
	\end{exercise}
	
	\begin{oss}
		Nel corso considereremo vero, anche se non lo dimostriamo, che somma e prodotto di funzioni misurabili è misurabile e massimo e minimo di funzioni misurabili è misurabile. Ne è immediata conseguenza che $ f^+ = max \{f,0\}$ e $f^-= min \{-f,0\}$ sono  misurabili , e che quindi lo è anche $\abs{f} = f^+ +f^- $.\\
		Ma se $|\abs{f}$ è misurabile, allora lo è anche $f$? In generale no, ad esempio se $ \exists A \subset X $ non misurabile , allora $X / A $ è non misurabile, dunque se $f=\rchi_A -\rchi_{A^C}$ non è misurabile, mentre $\abs{f} =1$ ed è misurabile.
	\end{oss}	
	
	
	\section{Funzioni semplici}
	
	\begin{definition}[Funzione semplice]
		Sia $(X,\mathscr{M})$ e sia $s(x)$ una funzione, si dice semplice se $\exists \alpha_1, \dots , \alpha_n \in \R$ e in $ A_1, \dots , A_n \mathscr{M} | A_i \cap A_j = \emptyset \forall i\neq j$ tali che $ s(x)= \sum^n_{i=1} \alpha_i \ \rchi_{A_i}(x)$.
	\end{definition}
	
	
	\begin{theorem}
		Sia $(X,\mathscr{M})$ e $f: X \to [0,\infty)$ funzione misurabile, allora $\exists \{f_n(x)\}$ successione monotona crescente di funzioni semplici positive tali che $f_n(x) \to f(x)$.
	\end{theorem}
	\begin{proof}
		Dividiamo in step la dimostrazione.
	\end{proof}
	
	\begin{definition}[Misura]
		Una funzione $\mu : \mathscr{M} \to [0,\infty]$ si dice misura se è $\sigma$ additiva.
	\end{definition}
	
	\begin{definition}[Spazio di misura]
		Sia $(X,\mathscr{M})$ uno spazio misurabile, se è dotato di misura $(X,\mathscr{M},\mu)$ si dice spazio di misura.
	\end{definition}
	
	\begin{prop} [Proprietà ]
		Sia
		\begin{enumerate}
			\item Se la misura è finita si ha $\mu (\emptyset)=0$;
			\item La misura è monotona;
			\item Siano $\{A_i\}_{i \in \N} $ insiemi misurabili, allora si ha che $ \mu(\bigcup_{i=1}^\infty) A_i \geq \sum_{i=1}^\infty\mu (A_i)$. Inoltr se la successione è crescente si ha $ \mu(\lim_n \bigcup_{i=1}^n) A_i = \sum^\infty_{i=1} \mu (A_i)$;
			\item Sia $\{C_i\}_{i \in \N} $ successione di insiemi misurabili tale che $ C_{i+1}\subset C_i \ , \ \mu (C_k)< \infty  $, allora $\mu(lim_n C_n) = lim_n \mu(C_n)$;
		\end{enumerate}
	\end{prop}
	
	\begin{proof}
		\begin{enumerate}
			\item Sia E un insieme misurabile , $ \mu (E) = \mu ( E\cup \emptyset )= \mu (E) + \mu (emptyset)\ \implies \mu( \emptyset )= 0$.
			\item Siano gli insiemi A e B tali che $A\subseteq B$, si ha che $B = A \cup (B/ A)$ dunque applicando al misura $\mu(B)=\mu(A)+\mu(B/A)\geq \mu (A)$.
			\item Dato che $A_i \subset \bigcup_{i=1}^\infty $ , applicando il punto precedente è immediato l'asserto. Considerando la successione $ C_1= A_1, \  C_2 = A_2\setminus A_1, \dots , C_n= A_n \setminus A_{n-1} $ tale che $C_i \cap C_j = \empty \forall i\ne j $ e si ha che $ \bigcap_i A_i = \bigcap_i C_i$. Dunque $\mu(\bigcap_{i=1}^\infty A_i) = \mu (\bigcap_{i=1}^\infty C_i) = \sum_{i=1}^\infty  = lim_n \sum_{i=1}^n \mu(C_i)= lim_n \mu (\bigcap_{i=1}^n C_i) = lim_n \mu (\bigcap_{i=1}^n A_i)$.
			\item La dimostrazione è analoga alla precedente . %% VORREI 		FARLA
		\end{enumerate}
	\end{proof}
	
	\begin{oss}
		A partire al $\delta $ di Dirac è possibile definire sulla $\sigma$-algebra di tutti gli insiemi un misura "ambigua": fissato $x_0 \in \R$ si ha che la misura $ \delta (E) = 1 $ se $x_0 \in E $ w $ \delta (E ) = 0 $ se $x_0 \ni E \quad \forall E \in \R$;
	\end{oss}
	
	\begin{definition} [Integrale di funzione semplice]
		SIa $f(x)= \sum_{i=1}^N \alpha_i \rchi_{A_i} $ una funzione semplice positiva  (con $\alpha_i \geq 0 \ \forall i \in I_N$) , si definisce l'integrale $\int_X f d\mu =  \sum_{i=1}^N \alpha_i \mu(A_i) \quad \forall A \in \mathscr(M)$.
	\end{definition}
	
	\begin{oss}
		Può accadere che, data la definizione dell'integrale delle funzioni semplici, compaia il prodotto $\mu(A) \alpha_i = \infty 0 $, tale condizione per convenzione si risolve ponendo  $ \infty 0 n= 0 $.
	\end{oss}
	
	\begin{definition} [Integrale di funzione misurabile]
		Sia $(X,\mathscr(M), \mu)$ uno spazio di misura, sia $f: X 	\to [0, \infty ]$ funzione misurabile, si definisce $\int_A f d\mu  = sup \{ \int_A s d\mu | s \text{ funzione semplice } f>s>0  \}$;
	\end{definition}
	
	\begin{prop}[Proprietà dell'integrale]
		L'integrale di funzioni misurabili su domini misurabili gode delle seguenti proprietà:
		\begin{enumerate}
			\item L'integrale gode della proprietà di monotonia rispetto alle funzioni;
			\item L'integrale gode della proprietà di monotonia rispetto al dominio di integrazione;
			\item Sia $f>\geq$ e sia $\infty>c\geq0$ una costante , allora $\int_A c d\mu = c \int_A d\mu  \quad \forall A \in \mathscr{M}$;
			\item Sia $f(x)=0 \ \forall x \in A$, allora $ \int_A f  d\mu = 0 $. 
			\item Se $\mu (E)=0 $ con  $ E \in \mathscr(E) $, allora $\int_E f d\mu = 0 \ \forall f$ ;
			\item Sia $f\geq 0 $ allora si ha che $ \int_E f d\mu = \int_X f \rchi_A$.
		\end{enumerate}
	\end{prop}
	
	\begin{definition} [Densità di misura]
		Sia $(X,\mathscr(M), \mu)$ uno spazio di misura, sia $f: X 	\to [0, \infty ]$ funzione misurabile, si definisce $\nu : \mathscr{M} \to [0,\infty ]$ densità della misura se $\nu (E)=\int_E f d\mu $ .
	\end{definition}
	
	\begin{oss}
		La densità di misura è a sua volta una misura  solo se f è misurabile.\\
	\end{oss}
	\begin{oss}[Non so neanche che sta dicendo]
		Sia $\gamma(E) \coloneq \int_E g d\mu $ con g misurabile, allora $(\gamma + \nu)(E)=\gamma(E)+\nu(E)= \int_E f+g d\mu$.
	\end{oss}
	
	\begin{theorem} [Teorema di convergenza monotona]
		Sia $(X,\mathscr(M),\mu )$ spazio di misura, sia $f_n$ una successione di funzioni misurabili tali che $ 0<f_1< \dots < f_n< \dots$ e $f_n: X \to [0,\infty ]$. Allora $\exists \lim_n f_(x) $, è misurabile, non negativo e si ha che $\int_X \lim_n f_n d \ \mu= \lim_n \int_X f_n d \ \mu $.
	\end{theorem}
	
	\begin{proof}
		Bisogna prima di tutto osservare che la successione delle $f_n$ converge in quanto monotona, e data la monotonia dell'integrale si ha che anche il limite della successione degli integrali converge. Indicando con $f \coloneq \lim_n f_n $ , si ha che $f$ è misurabile in quanto limite di una successione di funzioni misurabili ed è non negativa.\\
		Inoltre si ha che, dato che $\lim_m f_m \geq f_n \ \ forall n \in \N$, il $\lim_n \int_X f_n \ d\mu \leq \int_X \lim_n f_n \ d\mu $ dunque è necessario dimostrare solo l'altro verso.  A tale scopo studio diversi casi, partendo dal più semplice, per arrivare a generalizzare il risultato. \\
		Se $\lim_n f_n \ d\mu = \infty $, si avrà che data la disuguaglianza appena dimostrata dovranno necessariamente coincidere. \\
		Se $\lim_n f_n \ d\mu =L < \infty $, allora si ha che $\int_X f \ d\mu \geq L$, dunque considerando una funzione semplice $0<s<f$ e una costante $0<c<1$, si può definire la successione di insiemi $E_n = \{ x \in X | f_n(x) \geq c s(x)  \}$. Si ha che tale successione di insiemi è monotona crescente e per $n\to \infty$ forma un ricoprimento di X, dunque $X= \bigcup_n E_n $. 
		Quindi: $	\int_X f(x) \ d\mu > \int_{E_n} f_n(x) \ d\mu \geq \int_{E_n} c s(x) \ d\mu = c \int_{E_n} s(x) \  d\mu  	$
		Sia $	\nu(A) \coloneq \int_A s(x) \ d\mu 	$ una misura ( osservazione precedente ), riprendendo il passaggio precedente si ha $	\int_X f(x) \ d\mu >c \int_{E_n} s(x) \  d\mu	= c \nu(E_n )$ ; ma si ha che, per monotonia dell'integrale, $\nu(X) = \nu (\lim_n E_n )= \lim_n \nu(E_n )= \lim_n \int_{E_n} s(x) \ d\mu $. 
		Passando al limite si ha $\lim_n \int_X f_n (x) d\mu \geq c \lim_n \nu(E_n) = c \int_X  s(x) d\mu $, e infine passando al sup di s , si ha che : $	 \sup_s \lim_n \int_X f_n (x) d\mu = \lim_n \int_X f_n (x) d\mu \geq \sup_s c \int_X s(x) \ d\mu	= \int_X f(x) \ d\mu  $ passando al massimo su c, che è $c=1$, e portando dentro l'integrale il sup. 
	\end{proof}
	
	
	
\end{document}