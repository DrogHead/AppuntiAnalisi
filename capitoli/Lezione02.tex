\documentclass[../main.tex]{subfiles}

\begin{document}
	
	\begin{prop}[Misurabilità di liminf e limsup]
		Sia $(X,\mathscr{M}) $ spazio misurabile e	$\suc{f}{n}{\N}$ una successione di funzioni misurabili $f_n : X \to [-\infty, \infty]$. Allora $\liminf_n f_n$ e $\limsup_n f_n$ sono misurabili.
	\end{prop}
	\begin{proof}
		Per definizione, abbiamo che
		\begin{equation*}
			\limsup_n f_n = \inf_{n\in\N} \sup_{k\geq n} f_n \qquad \liminf_n f_n = \sup_{n\in\N} \inf_{k\geq n} f_n \,,
		\end{equation*}
		dunque occorre mostrare che $\sup_n f_n$ e $\inf_n f_n$ sono misurabili. Mostriamo la tesi per l'estremo superiore, lasciando al lettore la dimostrazione (analoga) per l'estremo inferiore. Per la proposizione precedente, detto $z(x) \defeq \sup_n f_n(x)$, mostriamo che $z^{-1}((a,+\infty))$ è misurabile per ogni $a\in \R$.
		\begin{align*}
			z^{-1}((a,+\infty)) &= \left\{x \in X \,\middle\vert\, z(x)\in(a,+\infty) \right\} \\
			&= \bigcup_{n\in\N} \left\{ x \in X \,\middle\vert\, f_n(x)\in(a,+\infty) \right\} \\
			&= \bigcup_{n\in\N} f_n^{-1}((a,+\infty)) \,.
		\end{align*}
		Essendo le $f_n$ misurabili, si trova la tesi.
	\end{proof}
	
	\begin{corol}
		Sia $(X,\mathscr{M}) $ spazio misurabile e	$\suc{f}{n}{\N}$ una successione di funzioni misurabili $f_n : X \to [-\infty, \infty]$. Se esiste, $\lim_n f_n$ è misurabile.
	\end{corol}
	
	\begin{exercise}
		Data una successione di funzioni misurabili $\suc{f}{n}{\N}$, l'insieme $A$ dei punti $x$ in cui esiste il limite puntuale $\lim_n f_n(x)$ è misurabile.
	\end{exercise}
	
	\begin{corol}
		Il massimo e il minimo di funzioni misurabili è ancora misurabile. Dunque, anche le funzioni parte positiva $f^+\defeq\max\{f,0\}$ e parte negativa $f^-\defeq\max\{-f,0\}$ sono misurabili.
	\end{corol}
	
	\begin{oss}
		Abbiamo osservato che, se $f$ è misurabile, allora anche $\abs{f}$ è misurabile, ma il viceversa è falso. Infatti, se $A\subseteq X$ è un insieme non misurabile, allora non sono misurabili neanche $A^C$, $\rchi_A$ e $\rchi_{A^C}$. La funzione $f=\rchi_A-\rchi_{A^C}$ è banalmente non misurabile, ma il suo valore assoluto è la funzione identicamente uguale a $1$, che invece è misurabile. 
	\end{oss}
		
	\begin{definition}[Funzione semplice]
		Una funzione $s : X \to \R$ si dice funzione semplice se si scrive nella forma
		\begin{equation*}
			s(x) = \sum_{i=1}^N c_i \rchi_{A_i} \,,
		\end{equation*}
		dove $c_i\in\R$ e $\ennu{A}{N}$ sono insiemi misurabili a due a due disgiunti.
	\end{definition}
	
	\begin{theorem}[Approssimazione di funzioni positive con funzioni semplici]
		Sia $(X,\mathscr{M})$ e $f: X \to [0,+\infty)$ funzione misurabile, allora $\exists \{f_n(x)\}$ successione monotona crescente di funzioni semplici positive tali che $f_n(x) \to f(x)$.
	\end{theorem}
	\begin{proof}
		La dimostrazione è esattamente uguale a quella fatta in Analisi III, riportiamo solo la costruzione.
		\begin{itemize}
			\item Per ogni $n\in\N$ e per ogni $k=1,\dots,n2^n$ si definiscono gli insiemi
			\begin{equation*}
				F_n = \{f\geq n\} \qquad E_{k,n} = \left\{\frac{k-1}{2^n} \leq f \leq \frac{k}{2^n}\right\}
			\end{equation*}	
			\item La successione di funzioni semplici
			\begin{equation*}
				s_n = n\rchi_{F_n} + \sum_{k=1}^{n2^n} \frac{k-1}{2^n} \rchi_{E_{k,n}}
			\end{equation*}
			è quella che verifica la tesi.
		\end{itemize}
		Si dimostra inoltre che, se $f$ è limitata, la convergenza di $s_n$ a $f$ è uniforme.
	\end{proof}
	
	\begin{definition}[Misura]
		Dato uno spazio misurabile $(X,\mathscr{M})$, una funzione $\mu : \mathscr{M} \to [0,+\infty]$ si dice misura se è $\sigma$-additiva, ossia se per ogni successione $\suc{E}{n}{\N}$ di insiemi misurabili a due a due disgiunti, risulta
		\begin{equation*}
			\mu\left(\bigcup_{n\in\N}E_n\right) = \sum_{n\in\N}\mu(E_n) \,.
		\end{equation*}
		La terna $(X,\mathscr{M},\mu)$ si dice ``spazio di misura''.
	\end{definition}
	
	\begin{prop}[Proprietà immediate di una misura]
		Sia $\mu$ una misura su $(X,\mathscr{M})$.
		\begin{enumerate}
			\item Se esiste $E\in\mathscr{M}$ tale che $\mu(E)<+\infty$ allora $\mu(\emptyset)=0$
			\item Se $A\subseteq B$ allora $\mu(A)\leq \mu(B)$
			\item Se $\suc{E}{n}{\N}$ è una successione crescente di misurabili si ha
			\begin{equation*}
				\mu\left(\bigcup_{n\in\N}E_n\right) = \lim_n \mu(E_n) \,.
			\end{equation*}
			\item Se $\suc{E}{n}{\N}$ è una successione decrescente di misurabili ed esiste un $E_N$ di misura finita si ha
			\begin{equation*}
				\mu\left(\bigcap_{n\in\N}E_n\right) = \lim_n \mu(E_n) \,.
			\end{equation*}
			\item $\mu$ è $\sigma$-subadditiva: per ogni successione $\suc{A}{n}{\N}$ di insiemi misurabili si ha
			\begin{equation*}
				\mu\left(\bigcup_{n\in\N}A_n\right) \leq \sum_{n\in\N}\mu(A_n) \,.
			\end{equation*}
		\end{enumerate}
	\end{prop}
	\begin{proof}
		La prima e la seconda proprietà seguono immediatamente dalla definizione, ricordando che $E=E\cup\emptyset$ e che $B=A\cup(B\setminus A)$, unioni disgiunte.
		\begin{enumerate}
			\item[3.] Definisco $C_1=E_1$, $C_2=E_2\setminus E_1$, e in generale $C_N=E_N\setminus E_{N-1}$, che sono insiemi disgiunti, e vale $\bigcup_n E_n = \bigcup_n C_n$. Allora si trova che
			\begin{align*}
				\mu\left(\bigcup_{n\in\N}E_n\right) &= \mu\left(\bigcup_{n\in\N}C_n\right) =
				\sum_{n=1}^\infty \mu(C_n) \\
				&= \lim_{N\to\infty} \sum_{n=1}^N \mu(C_n)
				= \lim_{N\to\infty} \mu\left(\bigcup_{n=1}^N C_n\right) \\
				&= \lim_{N\to\infty} \mu(E_N)
			\end{align*}
			\item[4.] WLOG sia $E_1$ l'insieme con misura finita. Allora definisco $D_n = E_1 \setminus E_n$ che è una successione crescente e trovo che
			\begin{align*}
				\mu\left(\bigcup_{n\in\N}D_n\right) &= \lim_{N\to\infty} \mu(D_N) \\
				\implies \mu\left(E_1\setminus \bigcap_{n\in\N}E_n \right) &= \lim_{N\to\infty} \mu(E_1)-\mu(E_N) \\
				\implies \mu(E_1) - \mu\left(\bigcap_{n\in\N}E_n\right) &= \mu(E_1) - \lim_{N\to\infty}\mu(E_N) \,,
			\end{align*}
			da cui segue la tesi.
			\item[5.] Mostriamo prima il caso finito. Per l'additività si ha che
			\begin{align*}
				\mu(A_1\cup A_2) &= \mu(A_1\setminus A_2) + \mu(A_2\setminus A_1) + \mu(A\cap A_2) \\
				&= \mu(A_1\setminus A_2) + \mu(A_1\cap A_2) + \mu(A_2\setminus A_1) + \mu(A\cap A_2) - \mu(A\cap A_2) \\
				&= \mu(A_1) + \mu(A_2) - \mu(A_1\cap A_2) \\
				&\leq \mu(A_1) + \mu(A_2) \,,
			\end{align*}
			e per induzione si estende il caso a ogni collezione finita $\ennu{A}{N}$. Per il caso infinito, definiamo la successione
			\begin{equation*}
				B_1 = A_1 \,, \quad B_N = \bigcup_{n=1}^N A_n \,, \quad \bigcup_{n\in\N}A_n = \bigcup_{N\in\N} B_N \,,
			\end{equation*}
			che è crescente. Allora si trova che
			\begin{align*}
				\mu\left(\bigcup_{n\in\N} A_n\right) &= \lim_{N\to\infty} \mu(B_N) = \lim_{N\to\infty} \mu\left(\bigcup_{n=1}^N A_n\right)
				\leq \lim_{N\to\infty} \sum_{n=1}^N \mu(A_n)
				= \sum_{n=1}^\infty \mu(A_n) \,.
			\end{align*}
		\end{enumerate}
	\end{proof}
	
	\begin{definition}[Delta di Dirac]
		Si dice ``delta di Dirac'' per un punto $x_0$ la misura definita nel seguente modo.
		\begin{equation*}
			\delta_{x_0}(E) = 
			\begin{cases}
				1 & \textup{se } x_0\in E \\
				0 & \textup{altrimenti.}
			\end{cases}
		\end{equation*}
	\end{definition}
	
	\begin{definition}[Integrale di funzione semplice non negativa]
		Consideriamo uno spazio di misura $(X,\mathscr{M},\mu)$ e una funzione semplice non negativa $s :X \to \R$. Si definisce l'integrale su $X$ di $s$ rispetto alla misura $\mu$ come
		\begin{equation*}
			\int_X s\dmu \defeq \sum_{i=1}^N c_i \mu(E_i) \,,
		\end{equation*}
		dove $c_i\in \R$ e $E_i \in \mathscr{M}$. Qualora il prodotto $c_i \mu(E_i)$ sia una forma indeterminata $0\cdot\infty$, per convenzione si pone tale prodotto a $0$.
	\end{definition}
	
	\begin{definition}[Integrale di funzione misurabile non negativa]
		Consideriamo uno spazio di misura $(X,\mathscr{M},\mu)$ e una funzione misurabile non negativa $f :X\to \R$. In virtù del teorema di approssimazione, si definisce l'integrale su $X$ di $f$ rispetto alla misura $\mu$ come
		\begin{equation*}
			\int_X f\dmu \defeq \sup\left\{ \int_X s\dmu \,\middle\vert\, s \textup{semplice, } 0\leq s\leq f \right\} \,.
		\end{equation*}
	\end{definition}
	
	\begin{nota}
		Se $E\subseteq X$ è misurabile, si definisce l'integrale di una funzione misurabile non negativa $f$ su $E$ rispetto alla misura $\mu$ come
		\begin{equation*}
			\int_E f\dmu \defeq \int_X f\rchi_E\dmu \,.
		\end{equation*}
	\end{nota}
	
	\begin{prop}[Proprietà dell'integrale]
		L'integrale appena definito gode delle seguenti proprietà:
		\begin{enumerate}
			\item Monotonia rispetto alle funzioni:
			\begin{equation*}
				f\leq g \implies \int_X f\dmu \leq \int_X g\dmu \,;
			\end{equation*}
			\item Monotonia rispetto all'insieme:
			\begin{equation*}
				A\subseteq B \implies \int_A f\dmu \leq \int_B f\dmu \,;
			\end{equation*}
			\item Omogeneità per costanti positive:
			\begin{equation*}
				\lambda \geq 0 \implies \int_X \lambda f\dmu = \lambda \int_X f\dmu \,;
			\end{equation*}
			\item Annullamento:
			\begin{align*}
				f(x) = 0 \,\forall x\in A \implies \int_A f\dmu = 0\\
				\mu(A) = 0 \implies \int_A f\dmu = 0 \,.
			\end{align*}
		\end{enumerate}
	\end{prop}
	
	\begin{definition}[Densità di misura]
		Consideriamo uno spazio di misura $(X,\mathscr{M},\mu)$ e una funzione misurabile non negativa $f :X\to\R$. La funzione d'insieme $\nu :\mathscr{M}\to[0,+\infty]$ si dice densità della misura $\mu$ se, per ogni $E\in\mathscr{M}$ si ha
		\begin{equation*}
			\nu(E) = \int_E f\dmu \,.
		\end{equation*}
		Talvolta questa condizione verrà indicata con $d\nu = fd\mu$. Osserviamo che la $\nu$ così definita è una misura se e solo se $f$ è misurabile. Quando $\nu$ è una misura, si ha che
		\begin{equation*}
			\int_X g\dnu = \int_X gf\dmu \,.
		\end{equation*}
	\end{definition}
	
	\begin{theorem} [Teorema di convergenza monotona]
		Sia $\suc{f}{n}{\N}$ una successione di funzioni misurabili non negative crescente in $n$, e sia $f$ il limite puntuale delle $f_n$. Allora $f$ è misurabile e si ha
		\begin{equation*}
			\int_X f\dmu = \lim_{n\to\infty}\int_X f_n\dmu
		\end{equation*}
	\end{theorem}
	\begin{proof}
		Il limite puntuale $f$ è banalmente misurabile essendo estremo superiore di funzioni misurabili. Inoltre, vale $f_n\leq f$ per ogni $n\in\N$. Definiamo la successione numerica $\suc{\alpha}{n}{\N}$ come
		\begin{equation*}
			\alpha_n = \int_X f_n \dmu \,,
		\end{equation*}
		che in virtù della monotonia dell'integrale è una successione crescente di numeri reali e in quanto tale ammette limite $\alpha$. Passando all'estremo superiore su $n$, si trova che
		\begin{equation*}
			\int_X f_n \dmu \leq \int_X f\dmu \implies \alpha \leq \int_X f\dmu \,.
		\end{equation*}
		Per mostrare la disuguaglianza opposta, sia $c\in(0,1)$ e consideriamo una funzione semplice tale misurabile tale che $0\leq s\leq f$. Definiamo quindi l'insieme
		\begin{equation*}
			E_n = \left\{ x\in X \,\middle\vert\, f_n\geq cs \right\} \,.
		\end{equation*}
		Si vede subito che $\suc{E}{n}{\N}$ è una successione crescente di insiemi misurabili e la loro unione è $X$. Infatti, se $x\in X$, abbiamo due casi:
		\begin{enumerate}
			\item Se $s(x)=0$ allora banalmente $x\in E_1$;
			\item Se $s(x)>0$ allora $cs(x)<s(x)\leq f(x)$ e quindi dal teorema di permanenza del segno, per $n$ sufficientemente grande si ha $cs(x)<f_n(x)$ da cui $x\in E_n$.
		\end{enumerate}
		Dunque, abbiamo
		\begin{equation*}
			\int_X f_n \dmu \geq \int_{E_n}f_n\dmu \geq \int_{E_n} cs\dmu = c\int_{E_n}s\dmu \,.
		\end{equation*}
		Passando al limite per $n\to\infty$ si ottiene
		\begin{equation*}
			\alpha \geq c\int_X s \,,
		\end{equation*}
		passando all'estremo superiore per $c\in(0,1)$ e successivamente all'estremo superiore per $s$ funzione semplice misurabile tale che $0\leq s\leq f$ si ottiene
		\begin{equation*}
			\alpha \geq \int_X f\dmu
		\end{equation*}
		che è la tesi.
	\end{proof}
	
	\begin{nota}
		Il teorema di convergenza monotona continua a valere nel caso in cui le ipotesi $f_n\leq f_{n+1}$ e $f=\lim_n f_n$ valgano quasi ovunque.
	\end{nota}
	
\end{document}