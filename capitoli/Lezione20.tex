\documentclass[../main.tex]{subfiles}

\begin{document}
	
	\begin{definition}[Insieme generico]
		Un sottoinsieme $Y\subseteq X$ si dice generico se il suo complemento è un insieme di prima categoria. Un insieme si dice di prima categoria se è unione numerabile di insiemi mai densi.
	\end{definition}
	
	\begin{prop}
		Se $X$ è uno spazio metrico completo e $f_n(x)$ è una successione di funzioni continue convergenti puntualmente a una funzione $f(x)$ in $X$, allora l'insieme dove $f$ è continua è generico.
	\end{prop}
	\begin{proof}
		Introduciamo la seguente notazione. Per ogni $x\in X$, per ogni $r>0$, definiamo
		\begin{equation*}
			\omega(f)(r,x) = \sup_{y,z\in B_r(x)}\abs{f(y)-f(z)} \,.
		\end{equation*}
		Osserviamo che $\omega$ è monotona crescente in $r$ e quindi esiste il suo limite per $r\to 0$: chiamiamo tale limite ``oscillazione di $f$'' e lo indichiamo con $\osc(f)(x)$. Notiamo immediataamente che $\osc(f)(x)=0\iff f$ è continua in $x$. Definiamo allora l'insieme
		\begin{equation*}
			E_\eps = \left\{x\in X \,\middle\vert\, \osc(f)(x)<\eps\right\}
		\end{equation*}
		è aperto per ogni $\eps>0$. Infatti, per ogni $x \in E_\eps$ esiste $r>0$ tale che $\omega(f)(r,x)<\eps$. Detta $B=B_{\frac{r}2}(x)$ e preso $x^*\in B$, si ha ancora che $\omega(f)(r,x^*)<\eps$ da cui $B\subseteq E_\eps$.
		
		Per concludere, mostriamo che
		\begin{equation*}
			\forall\eps>0 \,\forall B\subseteq X \,\exists B_{r_0}(x_0)\subseteq B \,\exists n\in\N \,: \abs{f_n(x)-f(x)}<\eps \,\forall x\in B_0 \,.
		\end{equation*}
		Consideriamo $\overline{B_1}\subset B$, che è uno spazio metrico completo (essendo un chiuso in uno spazio metrico completo), e definiamo
		\begin{equation*}
			W_\ell = \left\{x\in\overline{B_1} \,\middle\vert\, \sup_{j,k\geq\ell}\abs{f_j(x)-f_k(x)}\leq\eps \right\}
		\end{equation*}
		che è chiuso per ogni $\ell$, e inoltre $\overline{B_1}=\bigcup_{\ell\in\N} W_\ell$. Per il lemma di Baire, esiste $m\in\N$ tale che uno dei $W_m$ ha interno non vuoto, e quindi contiene una palla aperta $B_0$. Per costruzione, su $B_0$ l'estremo superiore definito in precedenza è $\leq\eps$. Nel limite per $k\to\infty$, la disuguaglianza diventa:
		\begin{equation*}
			\abs{f_j(x)-f(x)}\leq\eps \quad\quad \forall j\geq m
		\end{equation*}
		A patto di prendere $B_0$ di raggio sufficientemente piccolo, abbiamo che
		\begin{equation*}
			\abs{f(x)-f(z)}\leq\abs{f(x)-f_m(x)}+\abs{f_m(x)-f_m(z)}+\abs{f_m(z)-f(z)}
		\end{equation*}
		Il primo e il terzo termine sono $\leq\eps$, mentre per il termine centrale possiamo usare la continuità delle $f_m$ se $x,z\in B_0$ e maggiorarlo ancora una volta con $\eps$. Consideriamo ora gli insiemi
		\begin{equation*}
			F_n=\left\{x\in X \,\middle\vert\, \osc(f)(x)\geq \frac1n \right\}
		\end{equation*}
		ossia $F_n=E_\eps^C$ per $\eps=\frac1n$. L'insieme delle discontinuità di $f$ è l'unione di questi $F_n$, che sono chiusi mai densi. Infatti, se $F_n$ contiene una palla $B$ aperta, scelto $\eps=\frac1{4n}$, esiste $B_0\subset B$ tale che:
		\begin{equation*}
			\abs{f(y)-f(z)}\leq3\eps = \frac34 \frac1n < \frac1n
		\end{equation*}
		contro la definizione di $F_n$.
	\end{proof}
	
	\begin{corol}
		Consideriamo lo spazio di Banach $C[a,b]$ delle funzioni continue in un intervallo chiuso e limitato. Il sottoinsieme delle funzioni $f$ continue e mai differenziabili è generico.
		
		Un esempio di funzione mai differenziabili è la funzione di Weierstrass:
		\begin{equation*}
			f(x) = \sum_{n=0}^{+\infty} 2^{-n\alpha} e^{i2^n x} \quad\quad 0<\alpha<1
		\end{equation*}
		
		Questa serie converge uniformemente, ma la sua derivata, ottenuta derivando termine a termine
		\begin{equation*}
			f(x) = \sum_{n=0}^{+\infty} i 2^{n(1-\alpha)} e^{i2^n x} \quad\quad 0<\alpha<1
		\end{equation*}
		diverge in modulo per ogni $x$.
	\end{corol}
	
	\begin{lemma}
		Mostriamo l'equivalenza che servirà dopo.
	\end{lemma}
	\begin{proof}
		Sia $A$ aperto e sia $y_0\in\Lambda(A)$. Voglio dimostrare che esiste un $\eps>0$ tale che $B_Y(y_0,\eps)\subseteq\Lambda(A)$. Per suriettività esiste sicuramente $x_0\in A$ tale che $y_0=\Lambda x_0$. Dal momento che $A$ è aperto, contiene una palla $B_X(x_0,\delta)$, e abbiamo, anche usando la linearità, che
		\begin{equation*}
			\Lambda(B_X(0,\delta)) \supseteq B_Y(0,r\delta)
		\end{equation*}
		implica
		\begin{equation*}
			\Lambda(x_0+B_X(0,\delta)) = y_0+\Lambda(B_X(0,\delta)) \supseteq y_0+B_Y(0,r\delta) = B_Y(y_0,r\delta) \,.
		\end{equation*}
	\end{proof}
	
	\begin{theorem}[Teorema dell'applicazione aperta]
		Siano $X, Y$ spazi di Banach e sia $\Lambda : X\to Y$ operatore lineare. Se $\Lambda$ è continua e suriettiva, allora $\Lambda$ è aperta.
	\end{theorem}
	\begin{proof}
		Per il lemma precedente, è equivalente dimostrare che esiste $r>0$ tale che $\Lambda(B_X(0,1))\supseteq B_Y(0,r)$.
	
		Per suriettività si ha che 
		\begin{equation*}
			Y=\Lambda(X)=\bigcup_{n\in\N}\Lambda(B_X(0,n)) \,,
		\end{equation*}
		e quindi per il lemma di Baire esiste $n\in\N$ tale che $\overline{\Lambda(B_X(0,n))}$ ha interno non vuoto. Riscalando e usando la linearità, possiamo assumere $n=1$, e concludiamo che
		\begin{equation*}
			\exists y_0 \in Y \,: B_Y(y_0,\eps)\subseteq \overline{\Lambda(B_X(0,1))}
		\end{equation*}
		per qualche $\eps>0$. Quindi, per ogni $y\in B_Y(0,\eps)$ si ha che
		\begin{equation*}
			y_0+y\in\overline{\Lambda(B_X(0,1))} \,,
		\end{equation*}
		e quindi per disuguaglianza triangolare che
		\begin{equation*}
			y=y_0+(y-y_0) \in \overline{\Lambda(B_X(0,2))} \,.
		\end{equation*}
		Poiché è un chiuso, esiste una successione $\suc{x'}{n}{\N}\subseteq B_X(0,1)$ tale che $\Lambda x'_n \to y+y_0$, ed esiste un'altra successione $\suc{x''}{n}{\N}\subseteq B(0,1)$ tale che $\Lambda x''_n\to y_0$, e quindi la successione $x^*_n=x''_n-x'_n$ si trova che
		\begin{equation*}
			x^*_n \in B_X(0,2) \quad\,\quad \Lambda x^*_n \to y \in \overline{\Lambda(B_X(0,2))} \,.
		\end{equation*}
		In conclusione, abbiamo mostrato che
		\begin{equation*}
			B_Y(0,\eps) \subseteq \overline{\Lambda(B_X(0,2))} \iff B_Y\left(0,\frac{\eps}2\right)\subseteq\overline{\Lambda(B_X(0,1))}
		\end{equation*}
		Che è quanto volevamo mostrare. Essendo l'inclusione valida anche riscalando ambo i membri, per ogni $k\in\N$ si ha che
		\begin{equation*}
			B_Y(0,2^{-k}r)\subseteq\overline{\Lambda(B_X(0,2^{-k}))} \,,
		\end{equation*}
		e scegliendo $k=1$ troviamo il raggio $\frac{r}2$, esiste $x_1\in B_X\left(0,\frac12\right)$ tale che
		\begin{equation*}
			y-\Lambda x_1 \in B_Y\left(0,\frac{r}4\right) \,.
		\end{equation*}
		Proseguendo la costruzione, esiste $x_k\in B_X(0,2^{-k})$ tale che
		\begin{equation}
			y-\sum_{i=1}^k\Lambda x_i \in B_Y(0,2^{-(k+1)}r) \,.
		\end{equation}
		La successione $\sum_{k=1}^k x_k$ è convergente a un $x\in B_X(0,1)$ e quindi:
		\begin{equation}
			\norm*{\Lambda\sum_{i=1}^k x_i - y} \leq \frac{r}{2^{-(k+1)}} \xrightarrow{k\to\infty} 0
		\end{equation}
		Ossia per definizione e per continuità $\Lambda \sum_{i=1}^k x_k \to y$, che ci dà
		\begin{equation}
			B_Y\left(0,\frac{r}2\right) \subseteq \Lambda B_X(0,1)
		\end{equation}
	\end{proof}
	
	\begin{corol}
		Se $\Lambda$ è anche biettiva, allora $\Lambda^{-1}$ è continua. Ancora, se $\Lambda$ è biettiva, esistono due costanti $c_1,c_2>0$ tali che:
		\begin{equation}
			c_1 \norm{x}_X \leq \norm{\Lambda x}_Y \leq c_2 \norm{x}_X
		\end{equation}
	\end{corol}
	
	\begin{corol}
		Sia $X$ uno spazio di Banach con le norme $\norm{\cdot}_1$ e $\norm{\cdot}_2$. Allora è sufficiente che $\norm{\cdot}_2\leq c\norm{\cdot}_1$ per qualche $c\geq 0$ perché le due norme siano equivalenti.
	\end{corol}
	
	\begin{theorem}[Teorema del grafico chiuso]
		Se $X,Y$ sono spazi di Banach e $\Lambda : X\to Y$ è un operatore lineare. Allora $\Lambda$ è continua se e solo se è chiusa. 
	\end{theorem}
	
	\begin{nota}
		Ricordiamo che un'applicazione lineare si dice chiusa se il suo grafico $G_\Lambda=\{(x,y)\in X\times Y \,\vert\, y=\Lambda x\}$ è chiuso in $X\times Y$.
	\end{nota}
	
	\begin{proof}
		Osserviamo che l'implicazione $\implies$ è banale. Se $\Lambda$ è chiusa, allora il grafico $G_\Lambda$ è chiuso nella topologia di $X\times Y$ indotta dalla norma prodotto (somma delle due norme). Consideriamo quindi le proiezioni
		\begin{align}
			p_X : G_\Lambda \to X \quad&\quad p_X(x,y)=x \\
			p_Y : G_\Lambda \to Y \quad&\quad p_Y(x,y)=y
		\end{align}
		Tali proiezioni sono continue e lineari, e sono anche chiuse. In particolare, stando sul grafico, sono anche biezioni, e si ha $\Lambda=p_Y\circ p_X^{-1}$. Composizione di funzioni continue è continua.
	\end{proof}
	
	
	
	
	
\end{document}

















